\documentclass{llncs}

\usepackage{fullpage} % uses full 8.5x11 page

\renewcommand{\labelitemi}{$\bullet$}  % use normal bullets for lists

\title{Keyczar: A Cryptographic Toolkit}
\author{Arkajit Dey\inst{1}, Stephen Weis\inst{2} and Shawn
  Willden\inst{3} }

\institute{Massachusetts Institute of Technology, Cambridge, MA, USA 02139
\and
Google Inc., Mountain View, CA, USA 94043
\and
Google Inc., Boulder, CO, USA 80403}

\begin{document}
\maketitle

\begin{abstract}
Keyczar's goal is to make it easier for application developers to
safely use cryptography. Keyczar defaults to safe algorithms, key
lengths, and modes, and prevents developers from inadvertently
exposing key material. It uses a simple, extensible key versioning
system that allows developers to easily rotate and retire keys.
\end{abstract}

\section{Introduction and Philosophy}

The motivation for Keyczar grew out of a need to make cryptography
easier to use for developers. Developers improperly using cryptography
can create serious security vulnerabilities. For instance, developers
may use obsolete algorithms, weak key lengths, improper cipher modes,
or unsafely compose cryptographic operations. Another common developer
mistake is to fail to provision for key rotation or even to hard-code
keys in source code.

Keyczar's goal is to address these issues by providing a simple
application programming interface (API) for developers that handles
basic cryptographic details. Keyczar also provides a simple key
versioning and management system based on directories of
human-readable flat files, which will be refered to as
\emph{keysets}. More information about Keyczar is available from {\tt
  http://keyczar.org}.

\subsection{Keysets and Key Versioning}

A Keyczar keyset should be thought of as a single logical key that
contains some number of key versions. Since a keyset logically
represents a single key whose material evolves over time, but whose
purpose does not, the keyset also maintains a single copy of the key
metadata, defining usage, name, purpose, etc.  Key versions may have
different key material characteristics so that, for example, key size
may be increased over time.

Keyczar allows multiple key versions to be \emph{active} at once, to
ensure that messages received during a key rotation can be correctly
processed with no special work on the part of the recipient, but
allows only one key version to be \emph{primary}, to ensure that the
selection of the key version to use for creating new messages is
always unambiguous.

\section{Using KeyczarTool}\label{keytool}

All Keyczar keys are generated with the stand-alone KeyczarTool
utilities. Three implementations of KeyczarTool are available under
{\tt org.keyczar.KeyczarTool} in Java and {\tt keyczar.keyczart} in
Python and as {\tt keyczart} in C++.

\subsection{KeyczarTool create}
KeyczarTool must first create a new keyset using the {\tt create}
command. A newly created keyset will initially contain just a metadata
file, described in section \ref{metadata}.  A new keyset does not
contain any key versions, these are added with {\tt addkey}, described
in section \ref{addkey}.

{\tt KeyczarTool create} requires {\tt location} and {\tt purpose}
command-line flags that specify the location of the key set and its
purpose. Valid purposes are {\tt crypt} and {\tt sign}. The create
command may also take an optional {\tt name} flag to give a newly
created keyset a name. If the {\tt asymmetric} flag is specified, the
newly created set will contain asymmetric keys of the specified
algorithm. Currently DSA is supported for keysets with a sign
purpose. RSA is supported for both crypting and signing keysets.

Some example {\tt create} commands:
\begin{itemize}
\item Create a symmetric signing (HMAC) keyset: \\
{\tt KeyczarTool create --location=/path/to/keyset --purpose=sign}
\item Create a symmetric crypting (AES) keyset named ``Test'': \\
{\tt KeyczarTool create --location=/path/to/keyset --purpose=crypt --name=Test}
\item Create an asymmetric signing (DSA) keyset: \\
{\tt KeyczarTool create --location=/path/to/keyset --purpose=sign
--asymmetric=dsa}
\end{itemize}

\subsection{KeyczarTool addkey}\label{addkey}

All Keyczar keys are created using the {\tt addkey} command. This
command requires a keyset {\tt location} flag and may optionally have
{\tt status}, {\tt crypter} and {\tt size} flags. Section \ref{status}
describes the meaning of the status values, but briefly they are {\it
  primary}, {\it active}, and {\it inactive}. The default status is
{\it active}. User-specified key sizes are supported, although it is
recommended that only default or larger key sizes are used.

The {\tt addkey} command will create a new file in the keyset
directory with an integer version number that is one greater than the
currently largest version.  Version numbers start from 1 and are
described in Section \ref{versions}. For example, if the current
keyset contains the key file {\tt 1}, a new key version will be
created in the file {\tt 2}. Some example {\tt addkey} commands:


\begin{itemize}
\item Create a new primary key: \\
{\tt KeyczarTool addkey --location=/path/to/keyset --status=primary}
\item Create a new active key: \\
{\tt KeyczarTool addkey --location=/path/to/keyset} 
\end{itemize}

Keyczar supports encrypted keysets. The {\tt crypter} flag may be used
to encrypt a key set when adding a new key. It will specify the
location of an existing keyset which will be used to encrypt a newly
generated key:
 
\begin{itemize}
\item Create a new active key and encrypt it with another keyset: \\
{\tt KeyczarTool addkey --location=/path/to/keyset
--crypter=/path/to/crypting/keys}
\end{itemize}

\emph{Feature disparity:} The C++ implementation of Keyczar has an
additional feature, password-based encryption of keys.  This feature
is unwieldy in its current form, and not supported by the Java or
Python implementations and is therefore not recommended until the C++
implementation is improved and it is ported to Java and Python.

\subsection{KeyczarTool pubkey}

Public keys in Keyczar are exported from existing asymmetric key
sets. The {\tt pubkey} command requires both an existing {\tt
  location} flag and a {\tt destination} where public keys will be
exported. If the keyset under {\tt location} was not created with an
{\tt asymmetric} flag, then a {\tt pubkey} command will fail.

\begin{itemize}
\item Exptract a public key: \\
{\tt KeyczarTool pubkey --location=/path/to/keyset
  --destination=/path/to/dest}
\end{itemize}

\emph{Feature disparity:} The C++ implementation of Keyczar has an
additional feature, password-based encryption of keys.  This feature
is unwieldy in its current form, and not supported by the Java or
Python implementations and is therefore not recommended until the C++
implementation is improved and it is ported to Java and Python.

\subsection{KeyczarTool promote, demote, and revoke}

The {\tt promote}, {\tt demote}, and {\tt revoke} commands are used to
change key status values. Each of these commands require a {\tt
  location} and {\tt version} flag.

Promoting an {\it active} will raise its status to {\it primary}, and
promoting an {\it inactive} status will make it {\it active}. There
can only be a single {\it primary} key in a given key set, so if there
is an existing primary key it will be demoted to {\it active}.

Similarly, {\tt demote} will lower a {\it primary} key to {\it
  active}, and an {\it active} key to {\it inactive}. The {\tt revoke}
command will only work for {\it inactive} status values.  The {\tt
  revoke} command will permenantly delete key material, so should be
used with caution.

An {\it active} key version will be used by Keyczar if it is asked to
decrypt or verify a message encrypted or signed, respectively, by that
key version.  When Keyczar encrypts or signs a message, it will only
use the {\it primary} key version, if any.  If no key version has the
status {\it primary}, encryption or signing will fail.  Keyczar does
not use {\it inactive} keys at all, so if a message arrives that was
encrypted with a key marked {\it inactive}, decryption will fail.  As
already mentioned, {\it revoked} keys are deleted and therefore cannot
be used at all.

The purpose of these key version statuses is to facilitate smooth key
rotation.  Suppose that Alice regularly sends messages to Bob,
encrypted using Bob's public key, and that both are using Keyczar.
Suppose that Bob decides to rotate his public key.  The following
procedure will allow Alice and Bob to continue communicating without
interruption during the rotation process:

\begin{enumerate}
\item Bob creates a new key pair, by default marked {\it active}: \\
{\tt KeyczarTool create --location=bobskey --purpose=crypt}
\item Bob extracts the public key: \\
{\tt KeyczarTool pubkey --location=bobskey --destination=newpubkey}
\item Bob sends the new public key to Alice, perhaps in a new public
  key certificate.  Alice imports it into her {\tt bobskey} keyset.
\item Alice promotes the new key, (suppose it's version 2), to {\it
  primary}, which automatically demotes the old key to {\it active}:
  \\
{\tt KeyczarTool promote --location=bobskey --version=2}
\item Bob promotes the new key to {\it primary}, which automatically
  demotes the old key to {\it active}: \\
{\tt KeyczarTool promote --location=bobskey --version=2}
\item Bob demotes the old key to {\it inactive}: \\
{\tt KeyczarTool demote --location=bobskey --version=1}
\item Some time later, Bob revokes the old key, making it impossible
  ever to use it again: \\
{\tt KeyczarTool revoke --location=bobskey --version=1}
\end{enumerate}

After step 4, Alice's messages to Bob are encrypted with the new key,
but Bob doesn't have to perform step 5 at the same time, because he
still has the old key marked {\it active}, so Keyczar will still use
it..  As long as Bob doesn't take step 6 before Alice takes step 4, he
can continue decrypting her messages without interruption.

\subsection{KeyczarTool importkey and exportkey}

To enable interoperation with other cryptography libraries and tools,
and to facilitate exchanging keys between parties, KeyczarTool also
provides some commands for importing and exporting keys, using
standard formats.

The {\tt importkey} command is is used to import a public key
certificate in X.509 format\footnote{X.509 and PKCS\#8 files may be in
either PEM or DER format, KeyczarTool will detect which is used and
process it appropriately.  A ``DER'' file is an ASN.1 structure
conforming to some specification (e.g. X.509 certificate or PKCS\#8
private key) and encoded using the ASN.1 Distinguished Encoding
Rules.  A PEM file is a DER file which has been base64-encoded and
wrapped in an ASCII header and footer that provides a human-readable
description of the contents.}, or a public/private key pair,
in PKCS\#8 format. Into an existing keyset, which must already exist
and have the appropriate type and purpose.  PCKS\#8 files containing private keys are
generally password-protected, so {\tt importkey} allows a {\tt
passphrase} option to be specified.  To encrypt the destination
keyset, the {\tt crypter} option may be used to specify the location
of the keyset that will be used to encrypt the imported key.

\begin{itemize}
\item Import a public key from a certificate into an existing keyset:\\
{\tt KeyczarTool importkey --location=/path/to/keys --pemfile=/path/to/file.pem}
\end{itemize}

The {\tt exportkey} command is used to export the current primary
private key from a keyset, in PKCS\#8 format.  A passphrase must be
specified with the {\tt passphrase} option.

A common use of the {\tt exportkey} command is to export a key pair to
PKCS\#8 PEM format preparatory to generating a public key certificate.
Using openssl, the following steps may be used to create a public key
certificate request:

\begin{enumerate}
  \item Export the current primary key pair to a PKCS\#8 file:\\
    {\tt KeyczarTool exportkey --location=/path/to/keys  
      --dest=/path/to/file.pem \texttt{\char`\\} \\
      --passphrase=``mypass''}
  \item Generate a certificate signing request:\\
    {\tt openssl req -new -in /path/to/file.pem -out /path/to/file.csr}
\end{enumerate}

\section{Using Java Keyczar}

The {\tt org.keyczar} package contains four public classes that
developers will use for cryptographic basic functions: {\tt Crypter},
{\tt Encrypter}, {\tt Signer}, and {\tt Verifier}. All four classes
are children of the abstract {\tt Keyczar} class and may be
constructed either with a {\tt KeyczarReader} or a simple String
location of a keyset:


\begin{itemize}
  \item Constructor: {\tt Keyczar(KeyczarReader reader)}
  \item Constructor: {\tt Keyczar(String keySetLocation)}
\end{itemize}

Also, {\tt org.keyczar} contains four additional public classes that
developers will use for session encryption and decryption: {\tt
SessionEncrypter}, {\tt SessionDecrypter}, {\tt
  SignedSessionEncrypter}, and {\tt SignedSessionDecrypter}.  These
classes are not derived from {\tt Keyczar} and must be provided with
appropriate {\tt Keyczar}-derived class instances which provide the
basic cryptographic operations:

\begin{itemize}
  \item Constructor: SessionEncrypter(Encrypter encrypter)
  \item Constructor: SessionDecrypter(Crypter crypter)
  \item Constructor: SignedSessionEncrypter(Encrypter encrypter,
    Signer signer)
  \item Constructor: SignedSessionDecrypter(Crypter crypter,
    Verifier verifier)
\end{itemize}

Session encrypters and decrypters differ from Encrypters and Crypters
in that they do not encrypt or decrypt data directly with the key
provided.  Instead, data is encrypted or decrypted using a session
key, which is in turn protected by the provided key.  Session
encryption is recommended when using asymmetric keys.

\subsection{KeyczarReaders}

The {\tt KeyczarReader} interface may be implemented by any class that
reads key material from a keyset. Keyczar includes a {\tt
  KeyczarFileReader} class that is used by the {\tt Keyczar(String
  keySetLocation)} constructor to read files from the local
disk. Developers may implement their own {\tt KeyczarReaders} to read
from arbitrary sources and pass them to {\tt Keyczar} constructors.

\subsection{Encrypters}

The {\tt Encypter} class is only able to encrypt data. {\tt Encrypter}
objects will be initialized by passing it the location of a crypting
key set.  {\tt Encrypters} can encrypt plaintext encoded in UTF-8
Strings, byte arrays, or ByteBuffers. {\tt Encrypters} must be
initialized with a keyset containing a {\it primary} status key to
encrypt any data.
\begin{itemize}
  \item {\tt String encrypt(String input)}: Encrypt a string and
    return a WebSafeBase64 output ciphertext.
  \item {\tt byte[] encrypt(byte[] input)}: Encrypt a byte array and
    return the ciphertext as a byte array.
  \item {\tt int encrypt(ByteBuffer input, ByteBuffer output)}:
    Encrypt the contents of the input buffer and write the ciphertext
    to the output. Return the number of bytes written.
  \item {\tt int ciphertextSize(int inputLength)}: Return the size of
    the ciphertext that would result from encrypting an input of the
    given length.
\end{itemize}

\subsection{Crypters}

The {\tt Crypter} class is a child of {\tt Encrypter} and is also able
to decrypt data. {\tt Crypter} objects may only be initialized by
crypting key sets that contain keys able to decrypt, namely symmetric
keys or private keys. {\tt Crypters} cannot be initialized with public
keys. {\tt Crypters} can decrypt ciphertext in WebSafeBase64 encoded
Strings, byte arrays, or ByteBuffers.


\begin{itemize}
  \item {\tt String decrypt(String input)}: Decrypt a WebSafeBase64
    string and return a UTF-8 encoded plaintext output.
  \item {\tt byte[] decrypt(byte[] input)}: Decrypt a ciphertext byte
    array and return the plaintext as a byte array.
  \item {\tt int decrypt(ByteBuffer input, ByteBuffer output)}:
    Decrypt the input ByteBuffer and write its output into the given
    output ByteBuffer.  Return the number of bytes written.
\end{itemize}

\subsection{Verifiers}

The {\tt Verifier} class is only able to verify signatures. {\tt
Verifier} objects may be initialized by passing it the location of
any signing key set.  It can verify signatures in WebSafeBase64
format, as byte arrays, or in ByteBuffers:


\begin{itemize}
  \item {\tt boolean verify(String data, String signature)}: Verifies
    a WebSafeBase64 encoded signature on the given UTF-8 String of
    data. Returns a boolean representing whether the signature is
    valid.
  \item {\tt boolean verify(byte[] data, byte[] signature)}: Verifies
    a signature in a byte array on the given byte array of
    data. Returns a boolean representing whether the signature is
    valid.
  \item {\tt boolean verify(ByteBuffer data, ByteBuffer signature}:
    Verifies a signauture in a ByteBuffer on the given ByteBuffer of
    data. Returns a boolean representing whether the signature is
    valid.
  \item {\tt boolean attachedVerify(byte[] signedBlob, byte[] hidden)}:
    Verifies a signature when the data and signature have been
    {\it attached} and are both contained in the signedBlob.  The
    attached signature computation also includes additional {\tt
    hidden} data (usually called a {\it nonce} in cryptographic
    literature).  This hidden data must be provided for verification
    to succeed.
  \item {\tt byte[] getAttachedData(byte[] signedBlob, byte[] hidden)}:
    Verifies an attached signature (see previous method) and extracts
    and returns the data portion of {\tt signedBlob}.  Throws
    KeyczarException if signature verification fails.
  \item {\tt byte[] getAttachedDataWithoutVerifying(byte[]
    signedBlob)}: Extracts and returns the data portion of the blob,
    but does not check the signature.
\end{itemize}

\subsection{Signers}

The {\tt Signer} class is a child of the {\tt Verifier} class and is
also able to generate signatures by signing data. {\tt Signer} objects
may only be initialized by keysets that are able to sign, namely
symmetric keys and private keys. {\tt Signers} may not be initialized
by public keys and may sign data in UTF-8 Strings, byte arrays, or
ByteBuffers. {\tt Signers} must be initialized with a keyset
containing a {\it primary} status key to sign any data.
\begin{itemize}
  \item {\tt String sign(String data)}: Signs a UTF-8 string of data
    and returns a signatures as a WebSafeBase64 string.
  \item {\tt byte[] sign(byte[] data)}: Signs a byte array of data and
    returns a signature in a byte array.
  \item {\tt int sign(ByteBuffer data, ByteBuffer signature}: Signs a
    ByteBuffer of input and writes the signature to the output
    ByteBuffer.  Returns the number of bytes written.
  \item {\tt byte[] attachedSign(byte[] data, byte[] hidden)}: Signs a
    byte array of data and a byte array of ``hidden'' data and returns
    the data and a signature in one byte array.
\end{itemize}

\subsection{Example: Putting KeyczarTool and Java Keyczar Together}

The following command-line commands would create a new AES encrypting
key:


\begin{verbatim}
mkdir /aeskeys
KeyczarTool create --location=/aeskeys --purpose=crypt
KeyczarTool addkey --location=/aeskeys --status=primary
\end{verbatim}

Then within a Java program, the following code would encrypt a string
of data:


\begin{verbatim}
Crypter crypter = new Crypter(``/aeskeys'');
String ciphertext = crypter.encrypt(``Some data to encrypt'');
\end{verbatim}

This same ciphertext would be decrypted with a call to {\tt
Crypter.decrypt()}:

\begin{verbatim}
String plaintext = crypter.decrypt(ciphertext);
\end{verbatim}

Using public keys is similar:


\begin{verbatim}
mkdir /rsakeys
KeyczarTool create --location=/rsakeys --purpose=crypt --asymmetric
KeyczarTool addkey --location=/rsakeys --status=primary
mkdir /rsa-publickeys
KeyczarTool pubkey --location=/rsakeys --destination=/rsa-publickeys
\end{verbatim}

Then in Java, we can encrypt using the public key set, but need the
private key set to decrypt:

\begin{verbatim}
// Initialize an Encrypter with the public keys
Encypter encrypter = new Encrypter(``/rsa-publickeykeys'');
String ciphertext = encrypter.encrypt(``Some data to encrypt'');
... 
// Initialize a Crypter with the private keys
Crypter crypter = new Crypter(``/rsakeys'');
String plaintext = crypter.decrypt(ciphertext);
\end{verbatim}

Since RSA encryption is limited to blocks of data no larger than the
size of the key, and since for security reasons it's better to use RSA
to encrypt a random session key, we can use session encryption:

\begin{verbatim}
// Initialize a SessionEncrypter with the public keys
Encypter encrypter = new Encrypter(``/rsa-publickeykeys'');
SessionEncrypter sessionEncrypter = new SessionEncrypter(encrypter);
String session = sessionEncrypter.getSessionString();
String ciphertext = sessionEncrypter.encrypt(``Some data to encrypt'');
...
// Initialize a SessionDecrypter with the private keys
Crypter crypter = new Crypter(``/rsakeys'');
SessionDecrypter sessionDecrypter = new SessionDecrypter(crypter, session);
String plaintext = crypter.decrypt(ciphertext);
\end{verbatim}

Signing and verifying is similarily easy:
\begin{verbatim}
KeyczarTool create --location=/hmackeys --purpose=sign
KeyczarTool addkey --location=/hmackeys --status=primary
\end{verbatim}

In Java:
\begin{verbatim}
Signer signer = new Signer(``/hmackeys'');
String signature = signer.sign(``Some data to sign'');
boolean verified = signer.verify(``Some data to sign'', signature);
\end{verbatim}

See section \ref{walkthrough} for a more detailed walkthrough and
sample test vectors.

\section{Using Python Keyczar}

The API for Python is very similar to the one for Java. The same four
classes, {\tt Crypter}, {\tt Encrypter}, {\tt Signer}, and {\tt
  Verifier} are in the {\tt keyczar.keyczar} module and descend from
the abstract {\tt Keyczar} class.  They have the same use as their
corresponding classes in Java. As in Java, they can constructed from a
{\tt Reader} or from a string location of a keyset. The methods of
construction are slightly different:

\begin{itemize}
  \item Constructor: {\tt Keyczar(reader)}
  \item Static Method: \verb|Keyczar.Read(key_set_location)|
\end{itemize}

\subsection{Readers}

The abstract class {\tt Reader} can be extended by any class that
reads key material from a keyset. Python Keyczar supplies a {\tt
  FileReader} implementation that is used by the {Keyczar.Read} static
method to read key files from local disk. Just as in Java, developers
can create their own {\tt Readers} by extending from the base class.

\subsection{Differences from Java Keyczar}

The main four public classes don't treat strings, byte arrays, or btye
buffers differently. The just deal in strings which can either be
UTF-8 or byte strings. And the output is also always a WebSafeBase64
string, not a byte array.


\begin{itemize}
  \item {\tt Encrypter.Encrypt: data (string) -> ciphertext (Base64 string)}
  \item {\tt Crypter.Decrypt: ciphertext (Base64 string) -> data (string)}
  \item {\tt Signer.Sign: data (string) -> sig (Base64 string)}
  \item {\tt Verifier.Verify: data (string), sig (Base64 string) -> boolean}
\end{itemize}

\section{Using C++ Keyczar}

\section{Keys}

\subsection{Statuses}\label{status}

As described in Section \ref{keytool}, Keyczar keys may have one of three
status values: {\it primary}, {\it active}, {\it inactive}.
Keys with a primary status value are used to generate new cryptomaterial, that
is, signatures or ciphertexts. Keys of all status values may be used to verify
or decrypt legacy data, but are not used to generate new data.

The idea is that a key will start its life as a primary key and be used to
generate cryptographic output. That key can be rotated by generating a new
primary key value, and changing the existing primary key to be active. Active
keys will be kept around to verify and decrypt existing data, until they are
inactive. A key that is inactive is identical
to an active key, except its usage statistics are collected by default. When a
key that is inactive is no longer used, it can be safely
revoked and removed from a key set entirely.

\subsection{Key Metadata}\label{metadata}

Each Keyczar keyset contains a metadata file that contains information about
that set's name, purpose, key type, encrypted status, and a list of key version
information (see Section \ref{versions}). Valid purposes are encrypting, 
verifying, encrypting and decrypting, and signing and verifying. Having the
purpose value in the metadata prevents a key from being used for an 
inappropriate purpose. For example, an RSA key that is typically used for
encrypting shouldn't also be used for signing.

These purposes correspond to their respective public and symmetric/private
keys, and are usable by the Keyczar classes specified in the table below: 

\vspace*{3mm}
\begin{tabular}{ l | l | l }
{\bf Purpose} &  {\bf Public, Private, or Symmetric } & {\bf Class used by} \\
\hline Encrypting & Public & Encrypter or Crypter \\ \hline
Encrypting and Decrypting & Private or Symmetric & Crypter \\ \hline
Verifying & Public & Verifier or Signer \\ \hline 
Verifying and Signing & Private or Symmetric & Signer \\ \hline
\end{tabular}
\vspace*{3mm}

Each metadata file contains a specific key type which is the only type of key
that set can contain. Currently, the valid key types are AES, HMAC-SHA1, DSA
Private, DSA Public, RSA Private, and RSA Public.

It also contains a boolean flag indicating whether the key files are stored in
encrypted format.

The metadata will contain a list of version information, which is
described in Section \ref{versions}. See Section \ref{walkthrough} for a sample
JSON metadata file.

\subsection{Versions}\label{versions}

Each key metadata (see Section \ref{metadata}) contains a list of version
numbers corresponding to the keys contained in a given keyset. Each version
specifies an integral version number, a status of either {\it active}, {\it
primary}, or {\it inactive}, and a boolean specifying whether
the key is exportable outside of Keyczar.

By default, key version numbers start from 1 and increase monotonically.
However, key version numbers may be any positive, non-zero integral values.

\subsection{Output Headers} \label{header}

Each Keyczar output is tagged with a Keyczar-specific header. The header format
is as follows:

\vspace*{3mm}
\begin{tabular}{| l | l |}
\hline
Format & Key Hash \\ \hline
1 byte & 4 bytes \\ \hline
\end{tabular}
\vspace*{3mm}

See Section \ref{walkthrough} for a concrete example of an output header.

\subsubsection{Version Byte}

The first byte of all output is a single byte representing the version of
keyczar that generated it. The current version byte is 1. All other values are
reserved.

\subsubsection{Key hash}\label{keyhash}

Each header includes 4-byte truncated SHA-1 hash of the raw key
material itself. The following values will be hashed for each key type:
\begin{itemize}
  \item HMAC: 4-byte integer value representing the length of the HMAC key in
  bytes, followed by the actual HMAC key bytes.
  \item AES: 4-byte integer value representing the length of the AES key,
  followed by the acutal AES key bytes, followed by the 4-byte KeyHash of the
  attached HMAC key.
  \item DSA Public: 4-byte integer value representing the length of the X.509
  representation of this key, followed by the X.509 representation.
  \item DSA Private: Same as DSA Public.
  \item RSA Public: 4-byte integer value representing the length of the X.509
  representation of this key, followed by the X.509 representation.
  \item RSA Private: Same as RSA Public. 
\end{itemize}

The SHA-1 output on each of these values is truncated to the first four bytes.
For example: The truncated 4-byte truncated SHA-1 values:
\begin{enumerate}
  \item The 4-byte truncated hash of SHA-1("") = {\tt da39a3ee 5e6b4b0d 3255bfef
  95601890 afd80709} encoded in hexadecimal is {\tt da39a3ee.}
  \item The truncated hash of SHA-1("The quick brown fox jumps over the lazy
  dog") = {\tt 2fd4e1c6 7a2d28fc ed849ee1 bb76e739 1b93eb12} is
  {\tt 2fd4e1c6 }.
\end{enumerate}

\subsection{Key Formats}\label{formats}

All key files are kept in JSON format and have a ``size'' field indicating the
length of the key in bits. Additional fields are described below for each key
type.

\subsubsection{HMAC}

HMAC keys also have the additional ``hmacKeyString'' field which is a Base64
representation of the raw HMAC key bytes.

\subsubsection{AES}

AES keys have an ``aesKeyString'' field which is a Base64 representation of the
raw AES key bytes. They also have an attached HMAC key, which is kept in a
``hmacKey'' field. There is an additional ``mode'' field which currently only
supports ``CBC''.

\subsubsection{Public RSA}

Public RSA keys have an ``x509'' field containing the Base64 representation of
a X.509 public key certificate. 

\subsubsection{Private RSA}

Private RSA keys contain a public RSA key, and a ``pkcs8'' field containing a
Base64 representation of a PKCS8 encoded RSA private key. 

\subsubsection{Public DSA}

Public DSA keys have an ``x509'' field containing the Base64 representation of
a X.509 public key certificate. 

\subsubsection{Private DSA}

Private DSA keys contain a public DSA key, and a ``pkcs8'' field containing a
Base64 representation of a PKCS8 encoded DSA private key. 

\section{Output Formats}

\subsection{Signatures}\label{signatures}

A Keyczar signature consists of a header (described in Section \ref{header})
followed by a algorithm-dependent signature:

\vspace*{3mm}
\begin{tabular}{| c | c |}
\hline
Header & Sign(header $|$ M) \\
\hline 5 bytes & Algorithm-dependent length \\ \hline
\end{tabular}
\vspace*{3mm}

(Note: $|$ represents string concatenation.)

\vspace*{3mm}

For example, HMAC keys will produce a 20-byte output, so have the form:

\vspace*{3mm}
\begin{tabular}{| c | c |}
\hline
Header & HMAC-SHA1(header $|$ M) \\ \hline
5 bytes & 20 bytes \\ \hline
\end{tabular}
\vspace*{3mm}
\subsection{Ciphertext}

A Keyczar ciphertext contains a header (described in Section \ref{header}), an
IV (if needed), the algorithm-dependent ciphertext, and a signature on the
preceding fields (for symmetric keys only):

\vspace*{3mm}
\begin{tabular}{| c | c | c | c |}
\hline
Header & IV (if any) & Encrypt(M) & Sign(header $|$ IV $|$ M) (if any) \\ \hline
5 bytes & Algorithm-dependent & Algorithm-dependent & Algorithm-dependent \\
\hline
\end{tabular}
\vspace*{3mm}

For example, 128-bit AES keys (which have an attached HMAC key) used in CBC
mode would produce the following output:

\vspace*{3mm}
\begin{tabular}{| c | c | c | c |}
\hline
Header & IV & AES-CBC(M) & HMAC-SHA1(header $|$ IV $|$ M) \\ \hline
5 bytes & 16 bytes & Algorithm-dependent & 20 bytes \\
\hline
\end{tabular}
\vspace*{3mm}

In contrast, RSA ciphertexts neither use IVs nor are signed. Thus, the RSA
ciphertext format is:

\vspace*{3mm}
\begin{tabular}{| c | c | c | c |}
\hline
Header & RSA-OAEP(M)  \\ \hline
5 bytes & Single RSA block  \\
\hline
\end{tabular}
\vspace*{3mm}

\section{Algorithm Details}

These are the default key lengths, algorithms, and padding modes used:

\begin{itemize}
  \item HMAC: Default keys are 256 bits. SHA1 used as the hash
  algorithm.
  \item AES: Default keys are 128 bits. 192 and 256 bit keys are also supported.
  CBC mode with PKCS\#5 padding is used by default.
  \item DSA: DSA-SHA1 signing algorithm used by default. Default key size is
  1024 bits.
  \item RSA Encryption: RSA-OAEP encryption is used. Default key size is 2048
  bits. 1024, 768, and 512 bit keys are also supported.
  \item RSA Signatures: RSA-SHA1 signing is used. Default key size is 2048
  bits. 1024, 768, and 512 bit keys are also supported.
\end{itemize}


\section{Walkthrough and Sample Data}\label{walkthrough}

\subsection{Creating an HMAC Keyset}
A {\tt meta} file is created by a call to the {\tt KeyczarTool create} command.
The following example was created using the command: \\
{\tt KeyczarTool create --location=/path/to/keyset --name=test --purpose=sign}

\begin{verbatim}
{
 "name":"test",
 "purpose":"SIGN_AND_VERIFY",
 "type":"HMAC_SHA1",
 "versions":[],
 "encrypted":false
}
\end{verbatim}

\subsection{Generating a HMAC key}

The following command will create a new key in an existing keyset: \\
{\tt KeyczarTool addkey --location=/path/to/keyset}

This will add key version information to the existing {\tt meta} file: 
\begin{verbatim}
{
 "name":"test",
 "purpose":"SIGN_AND_VERIFY",
 "type":"HMAC_SHA1",
 "versions":[
    { "exportable":false,
      "status":"ACTIVE",
      "versionNumber":1}
    ],
 "encrypted":false
}
\end{verbatim}

It will also create a key version file named ``1'':
\begin{verbatim}
{
 "hmacKeyString":"9E7yslYyE4GHzq2WbppOOjGqBL7OGXxy5OsFSqBn2ao",
 "size":256
}
\end{verbatim}

\subsection{Promoting a Key}
Since no status was specified, key version 1 was created with an ``active''
status. That means it can be used for verifying signatures, but not generating
new signatures. Since we just created this key, there are no existing
signatures to verify. We have to promote it to a ``primary'' status before
it's useful: \\
{\tt KeyczarTool promote --location=/path/to/keyset --version=1}

This will just alter the meta file to reflect the promotion:
\begin{verbatim}
{
 "name":"test",
 "purpose":"SIGN_AND_VERIFY",
 "type":"HMAC_SHA1",
 "versions":[
    { 
      "exportable":false,
      "status":"PRIMARY",
      "versionNumber":1
    }
  ],
 "encrypted":false
}
\end{verbatim}

We can generate a second key version with a primary status as follows: \\
{\tt KeyczarTool addkey --location=/path/to/keyset --status=primary}

This create a new key version ``2'' with a primary status. Keysets can only have
one primary key at a time, so this automatically demotes key version ``1'' to
be active:
\begin{verbatim}
{
 "name":"test",
 "purpose":"SIGN_AND_VERIFY",
 "type":"HMAC_SHA1",
 "versions":[
    { 
      "exportable":false,
      "status":"ACTIVE",
      "versionNumber":1
    },
    { 
      "exportable":false,
      "status":"PRIMARY",
      "versionNumber":2
    }
  ],
 "encrypted":false
}
\end{verbatim}

This creates another key version file ``2'' with the following contents:
\begin{verbatim}
{
  "hmacKeyString":"-jkgTURAPoBr9SaQ5NLsFq2Xu5Z54RmnHYHlv1RVE7s",
  "size":256
}
\end{verbatim}

\subsection{Key Hash Values}

Note that the Base64 value of the ``hmacKeyString'' in the previous
section decodes to the following 32 bytes in hexadecimal format:
\begin{verbatim}
fa, 39, 20, 4d, 44, 40, 3e, 80,
6b, f5, 26, 90, e4, d2, ec, 16,
ad, 97, bb, 96, 79, e1, 19, a7,
1d, 81, e5, bf, 54, 55, 13, bb
\end{verbatim}

These key bytes will be hashed to obtain a key identifier. The raw key bytes
will be prefixed by a byte representation of the length in bytes. In this case,
the length is 32, so the following bytes (in hexadecimal format) will be
hashed with SHA-1:\\ {\tt  SHA-1(00, 00, 00, 20, fa, 39, \ldots, 13, bb)}

This produces the following 20-byte SHA1 output:
\begin{verbatim}
d8, 36, 36, 62, d0, 6d, cb, a7,
35, cd, 6c, 69, 33, 40, df, 87,
ce, da, d9, 8d
\end{verbatim}

This is truncated down to the four bytes {\tt d8, 36, 36, 62}, which is
this key's hash identifier. As described in Section \ref{header}, all output
will be prefixed with a version byte (currently equal to 1) and this identifier.

\subsection{Signing and Verifying} 

To sign the string ``Hello world'' with our newly generated keys in Java, we
would simply use the following two lines of code.

\begin{verbatim}
Signer signer = new Signer("/path/to/keyset"); 
String signature = signer.sign("Hello world");
\end{verbatim}

Keyczar will convert the string to the bytes to the following values:
\begin{verbatim}
48, 65, 6c, 6c, 6f, 20, 77, 6f, 72, 6c, 64
\end{verbatim} 

The output Base64 encoded signature would be the value {\tt
Adg2NmIPTshpioGdunRaGRYWFNgheKmjvg}, which corresponds to the bytes:
\begin{verbatim}
1, d8, 36, 36, 62, f, 4e, c8,
69, 8a, 81, 9d, ba, 74, 5a, 19,
16, 16, 14, d8, 21, 78, a9, a3,
be,
\end{verbatim}

Note that the first output byte is the version byte 1, followed by the 4-byte
key hash identifier, and a 20-byte HMAC-SHA1 output. All signatures produced by
this key will have this same identifier. Verification of this signature is
simple:
\begin{verbatim}
boolean isValid =
    signer.verify("Hello world", "Adg2NmIPTshpioGdunRaGRYWFNgheKmjvg");
\end{verbatim}

\section{Licenses and Dependencies}

Keyczar is available under an Apache 2.0 license \cite{apache2}. Java Keyczar
depends on the Google GSON package \cite{google-gson}. It also relies on the
Java's {\tt javax.crypto} package, which may not be available in all countries
due to local laws and regulations.

Python Keyczar depends on the Python Cryptography Toolkit \cite{python-crypto}
and simplejson \cite{simplejson}. 

\section{Acknowledgements}

Thanks to Ben Laurie, Marius Schilder, and Neil Daswani for their original
design contributions. Thanks to Sarvar Patel, Loren Kornfelder, Manuel Marquez
Garrido, and Laura Krotowski for their various contributions.

\bibliography{keyczar}
\bibliographystyle{acm}

\end{document}

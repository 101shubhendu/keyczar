%
% API Documentation for API Documentation
% Package keyczar
%
% Generated by epydoc 3.0.1
% [Mon Jul 21 20:04:50 2008]
%

%%%%%%%%%%%%%%%%%%%%%%%%%%%%%%%%%%%%%%%%%%%%%%%%%%%%%%%%%%%%%%%%%%%%%%%%%%%
%%                          Module Description                           %%
%%%%%%%%%%%%%%%%%%%%%%%%%%%%%%%%%%%%%%%%%%%%%%%%%%%%%%%%%%%%%%%%%%%%%%%%%%%

    \index{keyczar \textit{(package)}|(}
\section{Package keyczar}

    \label{keyczar}
Keyczar Cryptography Toolkit

Collection of tools for managing and using cryptographic keys. Goal is to 
make it easier for developers to use application-layer cryptography.

\textbf{Authors:}
arkajit.dey@gmail.com (Arkajit Dey),
    steveweis@gmail.com (Steve Weis)


%%%%%%%%%%%%%%%%%%%%%%%%%%%%%%%%%%%%%%%%%%%%%%%%%%%%%%%%%%%%%%%%%%%%%%%%%%%
%%                                Modules                                %%
%%%%%%%%%%%%%%%%%%%%%%%%%%%%%%%%%%%%%%%%%%%%%%%%%%%%%%%%%%%%%%%%%%%%%%%%%%%

\subsection{Modules}

\begin{itemize}
\setlength{\parskip}{0ex}
\item \textbf{errors}: Contains hierarchy of all possible exceptions thrown by Keyczar.



  \textit{(Section \ref{keyczar:errors}, p.~\pageref{keyczar:errors})}

\item \textbf{keyczar}: Collection of all Keyczar classes used to perform cryptographic functions: 
encrypt, decrypt, sign and verify.



  \textit{(Section \ref{keyczar:keyczar}, p.~\pageref{keyczar:keyczar})}

\item \textbf{keyczart}: Keyczart(ool) is a utility for creating and managing Keyczar keysets.



  \textit{(Section \ref{keyczar:keyczart}, p.~\pageref{keyczar:keyczart})}

\item \textbf{keydata}: Encodes the two classes storing data about keys:



  \textit{(Section \ref{keyczar:keydata}, p.~\pageref{keyczar:keydata})}

\item \textbf{keyinfo}: Defines several 'enums' encoding information about keys, such as type, 
status, purpose, and the cipher mode.



  \textit{(Section \ref{keyczar:keyinfo}, p.~\pageref{keyczar:keyinfo})}

\item \textbf{keys}: Represents cryptographic keys in Keyczar.



  \textit{(Section \ref{keyczar:keys}, p.~\pageref{keyczar:keys})}

\item \textbf{readers}: A Reader supports reading metadata and key info for key sets.



  \textit{(Section \ref{keyczar:readers}, p.~\pageref{keyczar:readers})}

\item \textbf{util}: Utility functions for keyczar package.



  \textit{(Section \ref{keyczar:util}, p.~\pageref{keyczar:util})}

\end{itemize}

    \index{keyczar \textit{(package)}|)}
